\documentclass[11pt]{article}
\usepackage[margin=1in]{geometry}

\usepackage{algorithm}
\usepackage{algorithmicx}
\usepackage{algpseudocode}

%opening
\title{TODO: Title Here}
\author{TODO: Group Number Here \\ 
\small{TODO: Names listed here}}

\begin{document}

\maketitle

(NOTE: this is just a sample outline, and you are free to deviate from this 
format).

TODO: brief introduction.

\section{Motivation}
What is the problem that this algorithm is trying to solve (note: you may want 
to borrow from your P-2 write-up).

\section{Algorithm}
Provide pseudocode for the algorithm that your team is investigating.  Provide 
pseudocode for the algorithm, and be sure to 

\begin{algorithm}\caption{\textsc{AwesomeAlgorithm}}
 \begin{algorithmic}[1]
   \State {\bf Input:} TODO:list input
   \State {\bf Output:} TODO:list output\\
   
   \State $k \gets \textsc{Random}(0,10)$
   \If{$n=0$}
	\While{$i < k$}
	\State $a[i] \gets b[i]$
        \EndWhile\\
~~~~~~\Return $a$
   \Else\\
   ~~~~~~\Return $b$
   \EndIf
 \end{algorithmic}
\end{algorithm}

\section{Analysis}
Optional: you are welcome to provide a brief proof of correctness or a 
running time analysis of the algorithm.  The difficulty of doing this will, of 
ocurse, depend on your algorithm.

\section{Discussion}
Conclude with a discussion.  Things that you might want to consider to put in 
the discussion (or maybe in their own sections) include: what are some variants 
of this algorithm? Where can this algorithm be applied?  What improvements can 
be made to this algorithm (perhaps commenting on experimental results of the 
original paper)?  Has there been follow-up work (trace the paper's citations in 
google schoolar!)

\end{document}
