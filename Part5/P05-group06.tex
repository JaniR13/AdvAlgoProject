\documentclass[11pt,twocolumn]{article}
\usepackage[margin=1in]{geometry}

%opening
\title{Part 5: Bayesian Optimization of the flexible Job Shop Scheduling Problem}
\author{Group 06 \\ 
\small Janette Rounds, \small David Rice, \small Mitch Vander Linden}

\begin{document}
	\section{Accomplishments}
	
	We have completed several tasks since selecting our ''+1 element'', many of which are not on the schedule we created in the last part of the project. First of all, we turned the pseudocode of our algorithm into a "plain-English" description. We also researched Particle Swarm Optimization and Bayesian Network Learning in order to more fully understand our algorithm. 
	
	Our next step involved having some group discussions about the example we wanted to use. We needed an example that could shed light on the problem our algorithm solves in addition to showcasing the algorithm itself. We eventually decided to use a kitchen as our example "shop" and the "Job" is making Thanksgiving dinner. As anyone who has attempted to make Thanksgiving dinner in a single kitchen knows, scheduling oven time and other preparation tasks is NP-Complete. 
	
	Finally, we needed to decide on our video format. Since none of us are actors, and we want the quality of our explanation to be high, we decided not to "act" out the process of preparing Thanksgiving dinner. We decided to use a slideshow as our graphics portion instead, simply to reduce the visual confusion. Most kitchens are very busy places and we could display information much more concisely with a slideshow. We plan to record a voiceover for sound. We have since begun working on the slideshow portion. 
	
	\section{Remaining Tasks}
	\begin{enumerate}
		\item Finish slideshow
		\item Plan slide transitions
		\item Write script
		\item Record voiceover
	\end{enumerate}
	\section{Challenges}
	\begin{enumerate}
		\item Describing Particle Swarm Optimization and Bayesian Network Learning to someone without a background 
	\end{enumerate}
\end{document}