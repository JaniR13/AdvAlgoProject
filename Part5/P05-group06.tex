\documentclass[11pt,twocolumn]{article}
\usepackage[margin=1in]{geometry}

%opening
\title{Part 5: Bayesian Optimization of the flexible Job Shop Scheduling Problem}
\author{Group 06 \\ 
\small Janette Rounds, \small David Rice, \small Mitch Vander Linden}

\begin{document}
	\maketitle
	\section{Accomplishments} 
	We have completed several tasks since selecting our ''+1 element'', many of which are not on the schedule we created in the last part of the project. First of all, we turned the pseudocode of our algorithm into a "plain-English" description. We also researched Particle Swarm Optimization and Bayesian Network Learning in order to more fully understand our algorithm. 
	
	Our next step involved having some group discussions about the example we wanted to use. We needed an example that could shed light on the problem our algorithm solves in addition to showcasing the algorithm itself. We eventually decided to use a kitchen as our example "shop" and the "Job" is making Thanksgiving dinner. As anyone who has attempted to make Thanksgiving dinner in a single kitchen knows, scheduling oven time and other preparation tasks is in NP. 
	
	Finally, we needed to decide on our video format. Since none of us are actors, and we want the quality of our explanation to be high, we decided not to "act" out the process of preparing Thanksgiving dinner. Additionally, most kitchens are very busy places. We decided to use a slideshow as our graphics portion instead, to reduce the visual confusion, and make sure the quality of our video is high. We plan to record a voiceover for sound. We have since begun working on the slideshow portion. 
	
	\section{Remaining Tasks}
	The remaining tasks are outlined in the following table:
	
	\begin{table}[h!]
	\centering
	\begin{tabular}{ |l | c | r|}
	  \hline
	  Date & Persona & Description \\
	  \hline
	  \hline
	  11/21 & David R. & Introduction Portion \\
	  \hline
	  11/27 & Janette R. & +1 Portion \\
	  \hline
	  11/28 & Mitch VL. & Analysis and Conclusion \\
	  \hline
	  11/30 & All & Record Voice-over/Slides \\
	  \hline
	  12/03 & All & Complete Editing \\
	  \hline
	  12/04 & All & Video Presentations \\
	  \hline
	\end{tabular}
	\end{table}
	
	The introduction, +1, and analysis portion will involve designing graphics for the slides, and that will be the most time consuming part of each section.  Each member needs to begin working on their task at least three days before its assigned due date to be able to create effective and well-designed slides.
	
	The script for the introduction portion is already completed and some work has been done on the +1 and Analysis scripts.  Recording the voice over and timing the slide transitions will be a rather easy task and can likely be done in just a few hours.
	
	\section{Challenges}

	The most notable challenge has been trying to describe the particle swarm optimization and selection through bayesian network structures to a viewer that has no idea how those techniques work.  The problem that the fJSP solves is quite easy to understand, but the approximation through machine learning approched used is far too involved to successfully describe in simple terms in one or two minutes.  We will still attempt to describe it briefly, but it will be a challenge.
	
	Furthermore, it took some thought to come up with a "shop scenario" that could be used to effectively describe the algorithm through example and be relevant to the problem trying to be solved.  We realized a large amount of scheduling and optimization goes in to making a Thanksgiving dinner such that it is indeed an NP (perhaps even NP-hard) problem.  There are many tasks to be scheduled on many machines that can be multi-purpose, so it is a relevant (and fun given the upcoming Thanksgiving holiday) problem that will explain the fJSP well. 
	
\end{document}