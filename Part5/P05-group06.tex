\documentclass[11pt,twocolumn]{article}
\usepackage[margin=1in]{geometry}

%opening
\title{Part 5: Bayesian Optimization of the flexible Job Shop Scheduling Problem}
\author{Group 06 \\ 
\small Janette Rounds, \small David Rice, \small Mitch Vander Linden}

\begin{document}
	\section{Accomplishments}
	\begin{enumerate}
		\item Completed Script Outline
		\item Decided on an example to use (kitchen, making Thanksgiving dinner)
		\item Decided on video format (slideshow with voiceover explanation)
		\item Begun work on slideshow
		\item Researched Particle Swarm Optimization
		\item Researched Bayesian Network Learning
	\end{enumerate}
	\section{Remaining Tasks}
	The remaining tasks are outlined in the following table:
	
	\begin{table}[h!]
	\centering
	\begin{tabular}{ |l | c | r|}
	  \hline
	  Date & Persona & Description \\
	  \hline
	  \hline
	  11/21 & David R. & Introduction Portion \\
	  \hline
	  11/27 & JR & +1 Portion \\
	  \hline
	  11/28 & MVL & Analysis and Conclusion \\
	  \hline
	  11/30 & All & Record Voice-over/Slides \\
	  \hline
	  12/03 & All & Complete Editing \\
	  \hline
	  12/04 & All & Video Presentations \\
	  \hline
	\end{tabular}
	\end{table}
	
	The introduction, +1, and analysis portion will involve designing graphics for the slides, and that will be the most time consuming part of each section.  Each memeber needs to begin working on their task at least three days before its assigned due date to be able to create effective and well-designed slides.
	
	The script for the introduction portion is already completed and some work has been done on the +1 and Analysis scripts.  Recording the voice over and timing the slide transitions will be a rather easy task and can likely be done in just a few hours.
	
	\section{Challenges}
	
	The most notable challenge has been trying to describe the particle swarm optimization and selection through bayesian network structures to a viewed that has no idea how those techniques work.  The problem that the fJSP solves is quite easy to understand, but the approximation through machine learning approched used is far too involved to successfully describe in simple terms in one or two minutes.  We will still attempt to describe it briefly, but it will be a challenge.
	
	Furthermore, it took some thought to come up with an "shop scenario" that could be used to effectively describe the algorithm through example and be relevant to the problem trying to be solved.  We realized a large amount of scheduling and optimization goes in to making a Thanksgiving dinner such that it is indeed an NP (perhaps even NP-hard) problem.  There are many tasks to be scheduled on many machines that can be multi-purpose, so it is a relevant (and fun given the upcoming Thanksgiving holiday) problem that will explain the fJSP well.
	
\end{document}