\documentclass[11pt,twocolumn]{article}
\usepackage[margin=1in]{geometry}

\usepackage{algorithm}
\usepackage{algorithmicx}
\usepackage{algpseudocode}

%opening
\title{Part 4: Bayesian Optimization of the flexible Job Shop Scheduling Problem}
\author{Group 06 \\ 
\small Janette Rounds, \small David Rice, \small Mitch Vander Linden}

\begin{document}

\maketitle


\section{Introduction}
The following algorithm attempts to find an optimized way to schedule the Job-shop Scheduling Problem (JSP) in a more flexible manner called the flexible Job-shop Scheduling Problem (fJSP). The algorithm takes advantage of hybrid evolutionary algorithms that use particle swarm optimization and bayesian network structuring to learn the optimal relationships between machine, task, and objective function. 

\section{Motivation}
The Job-Shop Scheduling Problem (JSP) is an NP-Hard problem in computer science\cite{cheng1996tutorial}. Let us imagine we have a set of machines and a set of jobs to be completed. Each job is a set of operations and the operation order is fixed. In the classical formulation of JSP, each operation has a fixed processing time, and a required machine. However, these authors used the flexible job-shop scheduling problem (fJSP) that assumes that one machine can perform multiple kinds of operation, and as such, there is no fixed processing time \cite{sun2015bayesian}. There are several potential constraints we could use including: a job does not visit the same machines twice; there are no precedence constraints among operations of different jobs; operations can not be interrupted; etc. 

Both JSP and fJSP are minimization problems, that is, we generally want to minimize the overall time to completion of all the jobs. As we stated before, there is no way to solve this problem in less than exponential time. Therefore, there are several algorithms we can use to approximate the solution. The authors combined Bayesian Optimization and Evolutionary Algorithms approaches to approximate a solution for the fJSP. 
\section{Algorithm}
The Hybrid Evolution Algorithm displayed above takes as input the Job-Shop problem data and constraints, and returns an approximation of the best solution to the Flexible Job-Shop Problem. The variable $t$ represents the $t^{th}$ generation of the solution set - it is initialized to 0 on line 3. Next, the population p($t$) is initialized to include a random set of candidate solutions to the problem. p($t$) is further divided into subgroups in a random fashion. After the population has been initialized, the main \textbf{while} loop is entered, which will iterate until the the termination condition has been satisfied based on the problem constraints.

At the start of each loop iteration, the population is evaluated using particle swarm optimization (PSO). Compared to a genetic algorithm, a PSO is more likely to include the optimal solution in the problem space, so the result has a higher probability to be accurate\cite{sun2015bayesian}. The best solution for the population is kept, and if it is a valid solution, the population grouping is adjust according to the Bayesian Network for each subgroup. The loop then repeats, evaluating the data based on the new subgrouping of the population. When the loop termination condition is met, the best solution is returned.

\section{+1 Options}
We have selected a complex algorithm that solves a complex set of problems. If we were to simply explain all of the equations in PSO and Bayesian Network Learning (parts of our algorithm), we would have a video much longer than 4-5 minutes. However, we can run through a simple example. This would give people a high-level understanding of our algorithm without having to delve into the theory of Particle Swarm Optimization for example. 

\section{Discussion}
Conclude with a discussion.  Things that you might want to consider to put in 
the discussion (or maybe in their own sections) include: What are the biggest 
challenges that you will face for the remainder of this project?  

\bibliographystyle{plain}
\bibliography{bibfilename} 

\newpage
\appendix
\section{Timeline}
The following are a list of tasks that we need to accomplish in order to 
complete our project:

\paragraph{Structure the video.} There are four tiers involved in the completion of this project. (1) Planning and outlining the necessary knowledge needed to give a good presentation on film. (2) The filming of the video.(3) Editing of the video. (4) Presentation of the video.

\paragraph{Decide elements needed.}  

The following table gives the timeline of how we plan to accomplish these tasks:

\begin{table}[h!]
\centering
\begin{tabular}{ |l | c | r|}
  \hline
  Date & Persona & Description \\
  \hline
  \hline
  11/16 & David R. & Introduction Portion \\
  \hline
  11/19 & All & +1 Portion \\
  \hline
  11/21 & MVL, JR & Analysis and Conclusion \\
  \hline
  11/23 & All & Film Video \\
  \hline
  11/28 & All & Complete Editing \\
  \hline
  11/30 & All & Video Presentations \\
  \hline
\end{tabular}
\end{table}

\paragraph{Timeline description}
The video will consist of three portions: Introduction, the plus one section, and analysis and conclusion. The induction portion should briefly give history, overview, and inention of the algorithm.  The plus one portion should give an example in simple terms of the algorithm in use and possibly go in to an alternate way to do the particle swarm optimization if time allows.  The analysis and conclusion will explain why the algorithm is useful and potential issues with its design.

Once the three portions have been planned and scripted out, filming, editing, and presenting will take place.

\end{document}
